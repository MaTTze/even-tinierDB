%%%%%%%%%%%%%%%%%%%%%%%%%%%%%%%%%%%%%%%%%
% Short Sectioned Assignment
% LaTeX Template
% Version 1.0 (5/5/12)
%
% This template has been downloaded from:
% http://www.LaTeXTemplates.com
%
% Original author:
% Frits Wenneker (http://www.howtotex.com)
%
% License:
% CC BY-NC-SA 3.0 (http://creativecommons.org/licenses/by-nc-sa/3.0/)
%
%%%%%%%%%%%%%%%%%%%%%%%%%%%%%%%%%%%%%%%%%

%----------------------------------------------------------------------------------------
%	PACKAGES AND OTHER DOCUMENT CONFIGURATIONS
%----------------------------------------------------------------------------------------
\PassOptionsToPackage{fleqn}{amsmath}
\documentclass[paper=a4, fontsize=11pt]{scrartcl} % A4 paper and 11pt font size

\usepackage[T1]{fontenc} % Use 8-bit encoding that has 256 glyphs
\usepackage{fourier} % Use the Adobe Utopia font for the document - comment this line to return to the LaTeX default
\usepackage[english]{babel} % English language/hyphenation
\usepackage{amsmath,amsfonts,amsthm} % Math packages
\usepackage{graphicx}
\usepackage{lipsum} % Used for inserting dummy 'Lorem ipsum' text into the template

\usepackage{sectsty} % Allows customizing section commands
\allsectionsfont{\normalfont\scshape} % Make all sections centered, the default font and small caps

\usepackage{enumerate}
\usepackage{tikz}
\usepackage{tikz-qtree}
\usetikzlibrary{automata,positioning}
\usepackage{latexsym}
\usepackage{float}
\usepackage{listings}
\usepackage{ulem}

\usepackage[compact]{titlesec}
\titlespacing{\section}{0pt}{*10}{*1} % Spacing between sections

\usepackage{fancyhdr} % Custom headers and footers
\pagestyle{fancyplain} % Makes all pages in the document conform to the custom headers and footers
\fancyhead{} % No page header - if you want one, create it in the same way as the footers below
\fancyfoot[L]{} % Empty left footer
\fancyfoot[C]{} % Empty center footer
\fancyfoot[R]{\thepage} % Page numbering for right footer
\renewcommand{\headrulewidth}{0pt} % Remove header underlines
\renewcommand{\footrulewidth}{0pt} % Remove footer underlines
\setlength{\headheight}{0pt} % Customize the height of the header
\usepackage{fullpage}

\usepackage{amsmath}
\setlength{\mathindent}{25.0pt}

\numberwithin{equation}{section} % Number equations within sections (i.e. 1.1, 1.2, 2.1, 2.2 instead of 1, 2, 3, 4)
\numberwithin{figure}{section} % Number figures within sections (i.e. 1.1, 1.2, 2.1, 2.2 instead of 1, 2, 3, 4)
\numberwithin{table}{section} % Number tables within sections (i.e. 1.1, 1.2, 2.1, 2.2 instead of 1, 2, 3, 4)

\setlength\parindent{0pt} % Removes all indentation from paragraphs - comment this line for an assignment with lots of text


%----------------------------------------------------------------------------------------
%	TITLE SECTION
%----------------------------------------------------------------------------------------

\newcommand{\horrule}[1]{\rule{\linewidth}{#1}} % Create horizontal rule command with 1 argument of height

\title{	
\normalfont \normalsize 
\textsc{Technische Universit\"at M\"unchen, Query Optimization} \\ [25pt] % Your university, school and/or department name(s)
\horrule{0.5pt} \\[0.4cm] % Thin top horizontal rule
\huge Exercise 4 \\ % The assignment title
\horrule{2pt} \\[0.5cm] % Thick bottom horizontal rule
}

\author{Matthias Gollwitzer, Jan Schalkamp} % Your name

\date{\normalsize\today} % Today's date or a custom date

\begin{document}
\lstset{language=SQL}
\maketitle % Print the title

%----------------------------------------------------------------------------------------
%	PROBLEM 1
%----------------------------------------------------------------------------------------

\section{Exercise 1}
\begin{figure}[H]
\centering
  \begin{tabular}{ | c | c | c |}
  	\hline
   	\multicolumn{3}{|c|}{DP Table} \\ 
	\hline
	Problem & Join Tree & Cost \\ \hline \hline
    	$\{A\}$ & A & 10 \\ \hline 
    	$\{B\}$ & B & 20 \\ \hline 
	$\{A,B\}$ & $B \Join A$ & $100$ \\ \hline 
	$\{C\}$ & C & 100 \\ \hline
	$\{A,C\}$ & $C \Join A$ & $1.000$ \\ \hline  
	$\{B, C\}$ & $C \Join B$ & $200$ \\ \hline 
	$\{A,B, C\}$ & $(C \Join B) \Join A$ & $1.200$ \\ \hline
	\sout{$\{A,B, C\}$} & \sout{$(C \Join A) \Join B$} & \sout{$2.000$} \\ \hline  
	$\{A,B, C\}$ & $(B \Join A) \Join C$ & $1.100$ \\ \hline 
  \end{tabular}
  \caption{Example relations}
  \label{relations}
 \end{figure}

\begin{figure}[H]
	\centering
		\begin{tikzpicture}[shorten >=1pt,node distance=2cm,auto]		
		
		\node[state] (r1) {$A$};
		\node[state] (r2) [right of=r1] {$B$};
		\node[state] (r3) [below of=r2] {$C$};



		\path[->]	
			
			(r1) edge [dotted] node [swap] {1} (r3)
			(r3) edge [dotted] node {} (r1)
			(r1) edge node {0.5} (r2)
			(r2) edge node {} (r1)
			(r2) edge node {0.1} (r3)
			(r3) edge node {} (r2)
			;
		\end{tikzpicture}
	\caption{Query graph with cardinalities}
	  \label{qgraph}
\end{figure}
\end{document}