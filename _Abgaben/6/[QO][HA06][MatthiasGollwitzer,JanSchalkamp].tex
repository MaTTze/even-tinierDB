%%%%%%%%%%%%%%%%%%%%%%%%%%%%%%%%%%%%%%%%%
% Short Sectioned Assignment
% LaTeX Template
% Version 1.0 (5/5/12)
%
% This template has been downloaded from:
% http://www.LaTeXTemplates.com
%
% Original author:
% Frits Wenneker (http://www.howtotex.com)
%
% License:
% CC BY-NC-SA 3.0 (http://creativecommons.org/licenses/by-nc-sa/3.0/)
%
%%%%%%%%%%%%%%%%%%%%%%%%%%%%%%%%%%%%%%%%%

%----------------------------------------------------------------------------------------
%	PACKAGES AND OTHER DOCUMENT CONFIGURATIONS
%----------------------------------------------------------------------------------------
\PassOptionsToPackage{fleqn}{amsmath}
\documentclass[paper=a4, fontsize=11pt]{scrartcl} % A4 paper and 11pt font size

\usepackage[T1]{fontenc} % Use 8-bit encoding that has 256 glyphs
\usepackage{fourier} % Use the Adobe Utopia font for the document - comment this line to return to the LaTeX default
\usepackage[english]{babel} % English language/hyphenation
\usepackage{amsmath,amsfonts,amsthm} % Math packages
\usepackage{graphicx}
\usepackage{lipsum} % Used for inserting dummy 'Lorem ipsum' text into the template

\usepackage{sectsty} % Allows customizing section commands
\allsectionsfont{\normalfont\scshape} % Make all sections centered, the default font and small caps

\usepackage{enumerate}
\usepackage{tikz}
\usepackage{tikz-qtree}
\usetikzlibrary{automata,positioning}
\usetikzlibrary{arrows,calc,fit}
\usepackage{latexsym}
\usepackage{float}
\usepackage{listings}
\usepackage{ulem}
\usepackage{amssymb}
\usepackage{longtable}
 \usepackage{array}

\usepackage[compact]{titlesec}
\titlespacing{\section}{0pt}{*10}{*1} % Spacing between sections

\usepackage{fancyhdr} % Custom headers and footers
\pagestyle{fancyplain} % Makes all pages in the document conform to the custom headers and footers
\fancyhead{} % No page header - if you want one, create it in the same way as the footers below
\fancyfoot[L]{} % Empty left footer
\fancyfoot[C]{} % Empty center footer
\fancyfoot[R]{\thepage} % Page numbering for right footer
\renewcommand{\headrulewidth}{0pt} % Remove header underlines
\renewcommand{\footrulewidth}{0pt} % Remove footer underlines
\setlength{\headheight}{0pt} % Customize the height of the header
\usepackage{fullpage}

\usepackage{amsmath}
\setlength{\mathindent}{25.0pt}

\numberwithin{equation}{section} % Number equations within sections (i.e. 1.1, 1.2, 2.1, 2.2 instead of 1, 2, 3, 4)
\numberwithin{figure}{section} % Number figures within sections (i.e. 1.1, 1.2, 2.1, 2.2 instead of 1, 2, 3, 4)
\numberwithin{table}{section} % Number tables within sections (i.e. 1.1, 1.2, 2.1, 2.2 instead of 1, 2, 3, 4)

\setlength\parindent{0pt} % Removes all indentation from paragraphs - comment this line for an assignment with lots of text
\def\ojoin{\setbox0=\hbox{$\Join$}%
  \rule[+0.095ex]{.3em}{.5pt}\llap{\rule[1.35ex]{0.3em}{0.5pt}}}
\def\leftouterjoin{\mathbin{\ojoin\mkern-5.8mu\bowtie}}
\def\rightouterjoin{\mathbin{\Join\mkern-5.8mu\ojoin}}
\def\fullouterjoin{\mathbin{\ojoin\mkern-5.8mu\Join\mkern-5.8mu\ojoin}}
%----------------------------------------------------------------------------------------
%	TITLE SECTION
%----------------------------------------------------------------------------------------

\newcommand{\horrule}[1]{\rule{\linewidth}{#1}} % Create horizontal rule command with 1 argument of height

\title{	
\normalfont \normalsize 
\textsc{Technische Universit\"at M\"unchen, Query Optimization} \\ [25pt] % Your university, school and/or department name(s)
\horrule{0.5pt} \\[0.4cm] % Thin top horizontal rule
\huge Exercise 6 \\ % The assignment title
\horrule{2pt} \\[0.5cm] % Thick bottom horizontal rule
}

\author{Matthias Gollwitzer, Jan Schalkamp} % Your name

\date{\normalsize\today} % Today's date or a custom date

\begin{document}
\lstset{language=SQL}
\maketitle % Print the title

%----------------------------------------------------------------------------------------
%	PROBLEM 1
%----------------------------------------------------------------------------------------

\section{Exercise 1}
With BFS result in following ordering of nodes: $R_0 < R_1 < R_3 < R_2 < R_4$. Figure \ref{ccp} shows the execution of the \textit{EnumerateCsg(G)} algorithm, and table \ref{cmp} shows the \textit{EnumerateCmp(G,S1)} algorithm. The stars in table \ref{cmp} denote the execution of \textit{EnumerateCsgRec()}. Empty rows depict multiple outputs for the step above them. (See the next pages)
\begin{figure}[H]
\centering
  \begin{tabular}{ | c | c | c | c |}
	\hline
	S & X & N & emit \\ \hline \hline
    	\{$R_4$\} & $\{R_0 .. R_4\}$ & $\emptyset$ & \{$R_4$\} \\ \hline 
  	\{$R_2$\} & $\{R_0, R_1, R_2, R_3\}$ & $\emptyset$ & \{$R_2$\} \\ \hline 
	\{$R_3$\} & $\{R_0, R_1, R_3\}$ & $\{R_2\}$ & \{$R_3$\} \\ \hline 
	& & & \{$R_2, R_3$\} \\ \hline
	$\{R_2, R_3\}$ & $\{R_0, R_1, R_2, R_3\}$ & $\emptyset$ & - \\ \hline 
	\{$R_1$\} & $\{R_0, R_1\}$ & $\{R_2, R_4\}$ & \{$R_1$\}\\ \hline 
	& & & \{$R_1, R_2$\} \\ \hline
	& & & \{$R_1, R_4$\} \\ \hline
	& & & \{$R_1, R_2, R_4$\} \\ \hline
	$\{R_1, R_2\}$ & $\{R_0, R_1, R_2, R_4\}$ & $\{R_3\}$ & \{$R_1, R_2, R_3$\} \\ \hline 
	\{$R_1, R_2, R_3$\} & $\{R_0, R_1, R_2, R_3, R_4\}$ & $\emptyset$ & - \\ \hline 
	 \{$R_1, R_4$\} & $\{R_0, R_1, R_2, R_4\}$ & $\emptyset$ & - \\ \hline 
	\{$R_1, R_2, R_4$\} & $\{R_0, R_1, R_2, R_4\}$ & $\{R_3\}$ & \{$R_1, R_2, R_3, R_4$\} \\ \hline 
	\{$R_1, R_2, R_3, R_4$\} & $\{R_0, R_1, R_2, R_3, R_4\}$ & $\emptyset$ & - \\ \hline 
	\{$R_0$\} & $\{R_0\}$ & $\{R_1, R_3\}$ & \{$R_0$\} \\ \hline 
	& & & \{$R_0, R_1$\} \\ \hline
	& & & \{$R_0, R_3$\} \\ \hline
	& & & \{$R_0, R_1, R_3$\} \\ \hline
	\{$R_0, R_1$\} & $\{R_0, R_1, R_3\}$ & $\{R_2, R_4\}$ & \{$R_0, R_1, R_2$\} \\ \hline 
	& & & \{$R_0, R_1, R_4$\} \\ \hline
	& & & \{$R_0, R_1, R_2, R_4$\} \\ \hline
	\{$R_0, R_1, R_2$\} & $\{R_0, R_1, R_2, R_3, R_4\}$ & $\emptyset$ &- \\ \hline 
	\{$R_0, R_1, R_4$\} & $\{R_0, R_1, R_2, R_3, R_4\}$ & $\emptyset$ &- \\ \hline 
	\{$R_0, R_1, R_2, R_4$\} & $\{R_0, R_1, R_2, R_3, R_4\}$ & $\emptyset$ &- \\ \hline 
	\{$R_0, R_3$\} & $\{R_0, R_1, R_3\}$ & $\{R_2\}$ & \{$R_0, R_2, R_3$\} \\ \hline 
	\{$R_0, R_2, R_3$\} & $\{R_0, R_1, R_2, R_3\}$ & $\emptyset$ &- \\ \hline 	
	\{$R_0, R_1, R_3$\} & $\{R_0, R_1, R_3\}$ & $\{R_4\}$ & \{$R_0, R_1, R_3, R_4$\} \\ \hline 
	\{$R_0, R_1, R_3, R_4$\} & $\{R_0, R_1, R_3, R_4\}$ & $\{R_2\}$ & \{$R_0, R_1, R_2, R_3, R_4$\} \\ \hline 
	\{$R_0, R_1, R_2, R_3, R_4$\} & $\{R_0, R_1, R_2, R_3, R_4\}$ & $\emptyset$ & - \\ \hline 	
  \end{tabular}
  \caption{Connected Subgraph Enumeration}
  \label{ccp}
 \end{figure}
 
 \pagebreak
 
\begin{longtable}{|*{4}{>{\centering\arraybackslash}p{3cm}|}}
	\hline
	S & X & N & emit \\ \hline \hline
    	\{$R_4$\} & $\{R_0 .. R_4\}$ & $\emptyset$ & - \\ \hline 
  	\{$R_2$\} & $\{R_0, R_1, R_2, R_3\}$ & $\emptyset$ & - \\ \hline 
	\{$R_3$\} & $\{R_0, R_1, R_3\}$ & $\{R_2\}$ & \{$R_2$\} \\ \hline 
	* \{$R_2$\} & $\{R_0, R_1, R_2, R_3\}$ & $\emptyset$ & - \\ \hline
	$\{R_2, R_3\}$ & $\{R_0, R_1, R_2, R_3\}$ & $\emptyset$ & - \\ \hline 
	\{$R_1$\} & $\{R_0, R_1\}$ & $\{R_2, R_4\}$ & \{$R_4$\}\\ \hline 
	& & & \{$R_2$\} \\ \hline
	* $\{R_4\}$ & $\{R_0, R_1, R_2, R_4\}$ & $\emptyset$ & - \\ \hline 
	* $\{R_2\}$ & $\{R_0, R_1, R_2\}$ & $\{R_3\}$ & $\{R_2, R_3\}$ \\ \hline 
	* $\{R_2, R_3\}$ & $\{R_0, R_1, R_2, R_3\}$ & $\emptyset$ & - \\ \hline 
	\{$R_1, R_2,$\} & $\{R_0, R_1, R_2\}$ & $\{R_3, R_4\}$ & $\{R_4\}$ \\ \hline 
	& & & \{$R_3$\} \\ \hline
	* $\{R_4\}$ & $\{R_0, R_1, R_2, R_3, R_4\}$ & $\emptyset$ & -\\ \hline 
	* $\{R_3\}$ & $\{R_0, R_1, R_2, R_3\}$ & $\emptyset$ & - \\ \hline 
	\{$R_1, R_2, R_3$\} & $\{R_0, R_1, R_2, R_3\}$ & $\{R_4\}$ & $\{R_4\}$ \\ \hline 
	* $\{R_4\}$ & $\{R_0, R_1, R_2, R_3, R_4\}$ & $\emptyset$ & -\\ \hline 
	\{$R_1, R_4$\} & $\{R_0, R_1, R_4\}$ & $\{R_2\}$ & $\{R_2\}$ \\ \hline 
	* $\{R_2\}$ & $\{R_0, R_1, R_2\}$ & $\{R_3\}$ & $\{R_2, R_3\}$ \\ \hline 
	* $\{R_2, R_3\}$ & $\{R_0, R_1, R_2, R_3, R_4\}$ & $\emptyset$ & - \\ \hline 
	\{$R_1, R_2, R_4$\} & $\{R_0, R_1, R_2, R_4\}$ & $\{R_3\}$ & \{$R_1, R_2, R_3, R_4$\} \\ \hline 
	* $\{R_3\}$ & $\{R_0, R_1, R_2, R_3, R_4\}$ & $\emptyset$ & - \\ \hline 
	\{$R_1, R_2, R_3, R_4$\} & $\{R_0, R_1, R_2, R_3, R_4\}$ & $\emptyset$ & - \\ \hline 
	\{$R_0$\} & $\{R_0\}$ & $\{R_1, R_3\}$ & \{$R_3$\} \\ \hline 
	& & & \{$R_1$\} \\ \hline
	* $\{R_3\}$ & $\{R_0, R_1, R_3\}$ & $\{R_2\}$ & $\{R_2, R_3\}$ \\ \hline
	* $\{R_2, R_3\}$ & $\{R_0, R_1, R_2, R_3\}$ & $\emptyset$ & - \\ \hline
	* $\{R_1\}$ & $\{R_0, R_1\}$ & $\{R_2, R_4\}$ & $\{R_1, R_2\}$ \\ \hline 
	& & & \{$R_1, R_4$\} \\ \hline
 	* $\{R_1, R_2\}$ & $\{R_0, R_1, R_2, R_4\}$ & $\{R_3\}$ & $\{R_1, R_2, R_3\}$ \\ \hline
	* $\{R_1, R_2, R_3\}$ & $\{R_0, R_1, R_2, R_3, R_4\}$ & $\emptyset$ & - \\ \hline
	* $\{R_1, R_4\}$ & $\{R_0, R_1, R_2, R_4\}$ & $\emptyset$ & - \\ \hline
	$\{R_0, R_1\}$ & $\{R_0, R_1\}$ & $\{R_2, R_3, R_4\}$ & $\{R_4\}$ \\ \hline
	& & & \{$R_2$\} \\ \hline
	& & & \{$R_3$\} \\ \hline
	* $\{R_4\}$ & $\{R_0, R_1, R_2, R_3, R_4\}$ & $\emptyset$ & -\\ \hline 
	* $\{R_2\}$ & $\{R_0, R_1, R_2, R_3\}$ & $\emptyset$ & -\\ \hline 
	* $\{R_3\}$ & $\{R_0, R_1, R_3\}$ & $\{R_2\}$ & $\{R_2, R_3\}$\\ \hline 
	* $\{R_2, R_3\}$ & $\{R_0, R_1, R_2, R_3\}$ & $\emptyset$ & -\\ \hline 
	\{$R_0, R_1, R_2$\} & $\{R_0, R_1, R_2\}$ & $\{R_3, R_4\}$ & \{$R_4$\} \\ \hline 
	& & & \{$R_3$\} \\ \hline
	* $\{R_4\}$ & $\{R_0, R_1, R_2, R_3, R_4\}$ & $\emptyset$ & -\\ \hline 
	* $\{R_3\}$ & $\{R_0, R_1, R_2, R_3\}$ & $\emptyset$ & -\\ \hline 
	\{$R_0, R_1, R_4$\} & $\{R_0, R_1, R_4\}$ & $\{R_2, R_3\}$ & $\{R_2\}$ \\ \hline 
	& & & \{$R_3$\} \\ \hline
	* $\{R_2\}$ & $\{R_0, R_1, R_2, R_3, R_4\}$ & $\emptyset$ & -\\ \hline 
	* $\{R_3\}$ & $\{R_0, R_1, R_3, R_4\}$ & $\{R_2\}$ & $\{R_2, R_3\}$\\ \hline 
	* $\{R_2, R_3\}$ & $\{R_0, R_1, R_2, R_3, R_4\}$ & $\emptyset$ & -\\ \hline 
	\{$R_0, R_1, R_2, R_4$\} & $\{R_0, R_1, R_2, R_4\}$ & $\{R_3\}$ & \{$R_3$\} \\ \hline 
	* $\{R_3\}$ & $\{R_0, R_1, R_2, R_3, R_4\}$ & $\emptyset$ & -\\ \hline 
	\{$R_0, R_3$\} & $\{R_0, R_3\}$ & $\{R_1, R_2\}$ & $\{R_2\}$ \\ \hline 
	& & & \{$R_1$\} \\ \hline
	* $\{R_2\}$ & $\{R_0, R_1, R_2, R_3\}$ & $\emptyset$ & -\\ \hline 
	* $\{R_1\}$ & $\{R_0, R_1, R_3\}$ & $\{R_2\}$ & $\{R_1,R_2\}$\\ \hline 
	\{$R_0, R_2, R_3$\} & $\{R_0, R_2, R_3\}$ & $\{R_1\}$ & $\{R_1\}$ \\ \hline 
	* $\{R_1\}$ & $\{R_0, R_1, R_2, R_3\}$ & $\{R_4\}$ & $\{R_1,R_4\}$\\ \hline
	* $\{R_1, R_4\}$ & $\{R_0, R_1, R_2, R_3, R_4\}$ & $\emptyset$ & -\\ \hline 
	$\{R_0, R_1, R_3\}$ & $\{R_0, R_1, R_3\}$ & $\{R_2, R_4\}$ & $\{R_4\}$\\ \hline 
	& & & \{$R_2$\} \\ \hline
	* $\{R_4\}$ & $\{R_0, R_1, R_2, R_3, R_4\}$ & $\emptyset$ & -\\ \hline
	* $\{R_2\}$ & $\{R_0, R_1, R_2, R_3\}$ & $\emptyset$ & -\\ \hline
	$\{R_0, R_1, R_3, R_4\}$ & $\{R_0, R_1, R_3\}$ & $\{R_2\}$ & $\{R_2\}$\\ \hline
	* $\{R_2\}$ & $\{R_0, R_1, R_2, R_3, R_4\}$ & $\emptyset$ & -\\ \hline 
	$\{R_0, R_1, R_2, R_3, R_4\}$ & $\{R_0, R_1, R_2, R_3, R_4\}$ & $\emptyset$ & -\\ \hline 	
 \caption{Enumerating Complementary Subgraphs}
 \label{cmp}
  \end{longtable}
  
  
  
  
\pagebreak



\section{Exercise 2}
Below $B()$ denotes the benefit function and all numbers $i$ beside the Joins depict the relation $R_i$.
Step 1 of the simplification shows that the relations $R_2$ and $R_3$ should be joined before joining them with $R_0$. Figure (\ref{simpl1}) shows the resulting query graph after this first step. The arrow determines the reading direction of the hyper edge. In this case: Join $R_2$ and $R_3$ before $R_1$.

\begin{align*}
\textbf{Step 1:} \\
B(0 \Join 1, 0 \Join 3) &= &\frac{C((0 \Join 1) \Join 3)}{C((0 \Join 3) \Join 1)}& =& \frac{20+2000}{1000+2000} &=& 0,673& \\ \\
B(1 \Join 0, 1 \Join 2) &= &\frac{C((1 \Join 0) \Join 2)}{C((1 \Join 2) \Join 0)}& =& \frac{20+100}{100+100} &=& 0,6&\\ \\
B(1 \Join 0, 1 \Join 4) &= &\frac{C((1 \Join 0) \Join 4)}{C((1 \Join 4) \Join 0)}& =& \frac{20+5000}{5000+5000} &=& 0,502&\\ \\
B(1 \Join 2, 1 \Join 4) &= &\frac{C((1 \Join 2) \Join 4)}{C((1 \Join 4) \Join 2)}& =& \frac{100+25000}{5000+25000} &=& 0,836&\\ \\
B(1 \Join 2, 2 \Join 3) &= &\frac{C((1 \Join 2) \Join 3)}{C((2 \Join 3) \Join 1)}& =& \frac{100+500}{250+500} &=& 0,8&\\ \\
B(0 \Join 3, 3 \Join 2) &= &\frac{C((0 \Join 3) \Join 2)}{C((2 \Join 3) \Join 0)}& =& \frac{1000+500}{250+500} &=& 2 &\\ \\
\end{align*}

\begin{figure}[H]
	\centering
		\tikzstyle{b} = [text centered, thick]
		%\tikzstyle{c} = [rectangle, draw]
		\tikzstyle{l} = [draw, -latex',thick]  

\begin{tikzpicture}[-,auto,scale=1.25]
    \node [b]    (r0) {$R_0$};
    \node [b]    (r1)   at ([shift={(1,0)}] r0) {$R_1$};
    \node [b]    (r2)   at ([shift={(0,-1)}] r1) {$R_2$};
    \node [b]    (r3)   at ([shift={(0,-1)}] r0) {$R_3$};
    \node [b]    (r4)   at ([shift={(1,0)}] r1) {$R_4$};
    \coordinate  (r0_r2)       at ($(r0)!0.5!(r2)$);  
    
    \draw[-] (r0) edge (r1);
    \draw[-] (r1) edge (r2);
    \draw[-] (r1) edge (r4);
    \draw[<-] (r0) edge (r0_r2);	
    \draw[-] (r0_r2) edge (r2);
    \draw[-] (r0_r2) edge (r3);

\end{tikzpicture}
	\caption{First step of simplification}
	  \label{simpl1}
\end{figure}

For Step 2 we re-use the old calculations except those affected by the new hyper edge. The resulting graph is shown in Figure (\ref{simpl2}). The benefit of joining relations $R_0$ and $R_1$ before $R_4$ is this step's biggest benefit, wherefore a hyper edge between those relations is introduced. 
\begin{align*}
\textbf{Step 2:} \\
B(0 \Join \{2,3\}, \{2,3\}\ \Join 1) &= &\frac{C((0 \Join \{2,3\}) \Join 1)}{C((1 \Join \{2,3\}) \Join 0)}& =& \frac{750+1000}{750+1000} &=& 1& \\ \\
B(0 \Join 1, 0 \Join \{2,3\}) &= &\frac{C((0 \Join 1) \Join \{2,3\})}{C((0 \Join \{2,3\}) \Join 1)}& =& \frac{20+1000}{750+1000} &=& 0,68&\\ \\
\end{align*}

\begin{figure}[H]
	\centering
		\tikzstyle{b} = [text centered, thick]
		%\tikzstyle{c} = [rectangle, draw]
		\tikzstyle{l} = [draw, -latex',thick]  

\begin{tikzpicture}[-,auto,scale=1.25]
    \node [b]    (r0) {$R_0$};
    \node [b]    (r1)   at ([shift={(1,0)}] r0) {$R_1$};
    \node [b]    (r2)   at ([shift={(0,-1)}] r1) {$R_2$};
    \node [b]    (r3)   at ([shift={(0,-1)}] r0) {$R_3$};
    \node [b]    (r4)   at ([shift={(1,0)}] r1) {$R_4$};
    \coordinate  (r0_r2)       at ($(r0)!0.5!(r2)$);  
    \coordinate  (r0_r1)       at ($(r0)!0.5!(r1)$);  
    \coordinate  (r14)       at ([shift={(0,0.5)}] r1);  
    
    \draw[-] (r0) edge (r1);
    \draw[-] (r1) edge (r2);
    \draw (r0) edge[<-] (r0_r2);	
    \draw[-] (r0_r2) edge (r2);
    \draw[-] (r0_r2) edge (r3);
    \draw[-] (r0_r1) edge (r14);
    \draw (r14) edge[->] (r4);

\end{tikzpicture}
	\caption{Second step of simplification}
	  \label{simpl2}
\end{figure}

\section{Exercise 3}
\begin{figure}[H]
\centering
	\begin{tikzpicture}
	\tikzset{level distance=40pt, sibling distance=10pt}
	
	\Tree
	[.{$\Join (1)$}
		[.{A} ]
		[.{$\Join (2)$}
			[.{B} ]
			[.{$\rightouterjoin (3)$}
				[.{$\Join (4)$}
					[.{C} ]
					[.{D} ]
				]
				[.{$\Join (5)$}
					[.{E} ]
					[.{F} ]
				]
			]
		]
	]
	\end{tikzpicture}
\caption{Given join tree}
  \label{jtree}
\end{figure}

\begin{figure}[H]
\centering
  \begin{tabular}{| c | c | c |}
	\hline
	Join & SES & TES \\ \hline
	1 & \{A,B\} & \{A,B,C,E,D\} \\ \hline
	2 & \{B,C\} & \{B,C,E,D\} \\ \hline
	3 & \{C,E\} & \{C,E,D\} \\ \hline
	4 & \{C,D\} & \{C,D\} \\ \hline
	5 & \{E,F\} & \{E,F\} \\ \hline
  \end{tabular}
  \caption{Syntactic- and total eligibility set}
  \label{relations}
 \end{figure}
 
 \begin{figure}[H]
	\centering
		\tikzstyle{b} = [text centered, thick]
		%\tikzstyle{c} = [rectangle, draw]
		\tikzstyle{l} = [draw, -latex',thick]  

\begin{tikzpicture}[-,auto,scale=1.25]
    \node [b]    (a) {A};
    \node [b]    (b)   at ([shift={(1,0)}] a) {B};
    \node [b]    (c)   at ([shift={(1,0)}] b) {C};
    \node [b]    (d)   at ([shift={(0,-1)}] c) {D};
    \node [b]    (e)   at ([shift={(1,0)}] c) {E};
    \node [b]    (f)   at ([shift={(1,0)}] e) {F};
    \coordinate  (ed)       at ($(e)!0.5!(d)$);  
    \coordinate  (eed)       at ($(e)!0.75!(ed)$); 
    \coordinate  (1)       at ([shift={(0,-1)}] eed);   
    \coordinate  (2)       at ([shift={(0,-1.37)}] b);   
    \coordinate  (3)       at ([shift={(0,-0.75)}] b);
    \coordinate  (4)       at ([shift={(0,-0.75)}] a);
    

    \draw[-] (a) edge (b);
    \draw[-] (b) edge (c);
    \draw[-] (c) edge (d);
    \draw[-] (c) edge (e);
    \draw[-] (e) edge (f);
    
    \draw[-] (ed) edge (c);
    \draw[-] (ed) edge (d);
    \draw (ed) edge[->] (e);
    
    \draw[-] (eed) edge (1);
    \draw[-] (1) edge (2);
    \draw (2) edge[->] (b);
    
    \draw[-] (3) edge (4);
    \draw (4) edge[->] (a);
    
\end{tikzpicture}
	\caption{DPhyp graph}
	  \label{simpl2}
\end{figure}
\end{document}