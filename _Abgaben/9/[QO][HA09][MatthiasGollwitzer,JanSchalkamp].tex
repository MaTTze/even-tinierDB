%%%%%%%%%%%%%%%%%%%%%%%%%%%%%%%%%%%%%%%%%
% Short Sectioned Assignment
% LaTeX Template
% Version 1.0 (5/5/12)
%
% This template has been downloaded from:
% http://www.LaTeXTemplates.com
%
% Original author:
% Frits Wenneker (http://www.howtotex.com)
%
% License:
% CC BY-NC-SA 3.0 (http://creativecommons.org/licenses/by-nc-sa/3.0/)
%
%%%%%%%%%%%%%%%%%%%%%%%%%%%%%%%%%%%%%%%%%

%----------------------------------------------------------------------------------------
%	PACKAGES AND OTHER DOCUMENT CONFIGURATIONS
%----------------------------------------------------------------------------------------
\PassOptionsToPackage{fleqn}{amsmath}
\documentclass[paper=a4, fontsize=11pt]{scrartcl} % A4 paper and 11pt font size

\usepackage[T1]{fontenc} % Use 8-bit encoding that has 256 glyphs
\usepackage{fourier} % Use the Adobe Utopia font for the document - comment this line to return to the LaTeX default
\usepackage[english]{babel} % English language/hyphenation
\usepackage{amsmath,amsfonts,amsthm} % Math packages
\usepackage{graphicx}
\usepackage{lipsum} % Used for inserting dummy 'Lorem ipsum' text into the template

\usepackage{sectsty} % Allows customizing section commands
\allsectionsfont{\normalfont\scshape} % Make all sections centered, the default font and small caps

\usepackage{enumerate}
\usepackage{tikz}
\usepackage{tikz-qtree}
\usetikzlibrary{automata,positioning}
\usetikzlibrary{arrows,calc,fit}
\usepackage{latexsym}
\usepackage{float}
\usepackage{listings}
\usepackage{ulem}
\usepackage{amssymb}
\usepackage{longtable}
 \usepackage{array}

\usepackage[compact]{titlesec}
\titlespacing{\section}{0pt}{*10}{*1} % Spacing between sections

\usepackage{fancyhdr} % Custom headers and footers
\pagestyle{fancyplain} % Makes all pages in the document conform to the custom headers and footers
\fancyhead{} % No page header - if you want one, create it in the same way as the footers below
\fancyfoot[L]{} % Empty left footer
\fancyfoot[C]{} % Empty center footer
\fancyfoot[R]{\thepage} % Page numbering for right footer
\renewcommand{\headrulewidth}{0pt} % Remove header underlines
\renewcommand{\footrulewidth}{0pt} % Remove footer underlines
\setlength{\headheight}{0pt} % Customize the height of the header
\usepackage{fullpage}
\usepackage{hyperref}
\usepackage{pgfplots}


\usepackage{amsmath}
\setlength{\mathindent}{25.0pt}

\numberwithin{equation}{section} % Number equations within sections (i.e. 1.1, 1.2, 2.1, 2.2 instead of 1, 2, 3, 4)
\numberwithin{figure}{section} % Number figures within sections (i.e. 1.1, 1.2, 2.1, 2.2 instead of 1, 2, 3, 4)
\numberwithin{table}{section} % Number tables within sections (i.e. 1.1, 1.2, 2.1, 2.2 instead of 1, 2, 3, 4)

\setlength\parindent{0pt} % Removes all indentation from paragraphs - comment this line for an assignment with lots of text
\def\ojoin{\setbox0=\hbox{$\Join$}%
  \rule[+0.095ex]{.3em}{.5pt}\llap{\rule[1.35ex]{0.3em}{0.5pt}}}
\def\leftouterjoin{\mathbin{\ojoin\mkern-5.8mu\bowtie}}
\def\rightouterjoin{\mathbin{\Join\mkern-5.8mu\ojoin}}
\def\fullouterjoin{\mathbin{\ojoin\mkern-5.8mu\Join\mkern-5.8mu\ojoin}}
%----------------------------------------------------------------------------------------
%	TITLE SECTION
%----------------------------------------------------------------------------------------

\newcommand{\horrule}[1]{\rule{\linewidth}{#1}} % Create horizontal rule command with 1 argument of height

\title{	
\normalfont \normalsize 
\textsc{Technische Universit\"at M\"unchen, Query Optimization} \\ [25pt] % Your university, school and/or department name(s)
\horrule{0.5pt} \\[0.4cm] % Thin top horizontal rule
\huge Exercise 9 \\ % The assignment title
\horrule{2pt} \\[0.5cm] % Thick bottom horizontal rule
}

\author{Matthias Gollwitzer, Jan Schalkamp} % Your name

\date{\normalsize\today} % Today's date or a custom date

\begin{document}
\lstset{language=SQL}
\maketitle % Print the title

%----------------------------------------------------------------------------------------
%	PROBLEM 1
%----------------------------------------------------------------------------------------

\section{Exercise 1}
	\begin{align*}
		&  & n \choose (n-k) &= \frac{n!}{(n-(n-k))! (n-k)!}\\
		& & &= \frac{n!}{(n-n+k)! (n-k)!}\\
		& & &= \frac{n!}{k! (n-k)!}\\
		& & &= \frac{n!}{(n-k)! k!}\\
		& & &= {n \choose k}
	\end{align*}
	\hfill $\Box$\\
	
	
\section{Exercise 2}
	\begin{align*}
		|R| = N = 6 &  & \\
		||R|| = m = 3& & &\\
		\text{tupels/page} = n = 2 & & & &\\
		|\text{requested tuples}| = k = 2 & & & &\\
		& P_1 &= {N \choose k} = {6 \choose 2}  &= 15 &\\
		& P_2 &  &= 3 &\\
		& p & &= \frac{3}{15} &\\
		& \text{accesses}_{\text{avg}} &= 1*p + 2*(1-p) = \frac{3}{15} + \frac{24}{15} &= 1.8 &
	\end{align*}
	With $P_1$ and $P_2$ being the number of ways to pick two tuples and to pick both tuples from a single bucket. Hence, $p$ determines the possibility of picking a bucket with both requested tuples inside and $(1-p)$ being the possibility of having to look in two different buckets (note, that there aren't any more options, as we are only looking for two tuples). Therefore, the average number of page accesses is $\text{accesses}_{\text{avg}} = 1.8$. It basically is $\textit{number of accessed pages}*\textit{possibility to access this number of pages}$. To show that it yields the correct solution, we evaluate Yao's formula using the same parameters as above:
	\begin{align*}
		& & \bar{y}^{N, m}_n (k)&  &=& & m*y^{N,m}_n (k) &\\
		& \Longrightarrow & \bar{y}^{6, 3}_2 (2)& &=& &3*y^{6,3}_2 (2) &= \\
		&= & & 3*\left( 1-\frac{{(6-2) \choose 2}}{{6 \choose 2}}\right) & &=& 3*\left(1-\frac{{4 \choose 2}}{{6 \choose 2}}\right)  &= \\
		&= & & 3*0.8 & &=& 1.8  &
	\end{align*}
	
	
\section{Exercise 3}
\center
\begin{tikzpicture}
	\begin{axis}[width=\textwidth-2cm, every axis plot post/.append style={
  		mark=none,
  		%domain=0:1000,
  		samples=1000,
  		smooth}, % All plots: 50 samples, smooth, no marks
		legend style={legend pos = south east},
		xlabel={Number of requested tuples},
		ylabel={Expected number accessed pages},
  		xmin=0,
  		xmax=1000,
  		ymin=0,
  		ymax=100,
  		enlargelimits=upper] % extend the axes a bit to the right and top
 
		\addplot[blue, domain=0:1000] {100* (1-(1-x/1000)^10)};
		\addlegendentry{Waters}
		\addplot[red, domain=0:1000] {100 * (1-((990!*(1000-x)!)/(1000!*(990-x)!))};
		\addlegendentry{Yao}
		\addplot[green, domain=0:50] {x};
		\addlegendentry{Bernstein}
		\addplot[green, domain=50:200]{(x+100)/3};
		\addplot[green, domain=200:1000]{100};
		
		
	\end{axis}
\end{tikzpicture}
	

\end{document}