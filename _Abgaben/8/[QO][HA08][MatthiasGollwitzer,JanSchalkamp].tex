%%%%%%%%%%%%%%%%%%%%%%%%%%%%%%%%%%%%%%%%%
% Short Sectioned Assignment
% LaTeX Template
% Version 1.0 (5/5/12)
%
% This template has been downloaded from:
% http://www.LaTeXTemplates.com
%
% Original author:
% Frits Wenneker (http://www.howtotex.com)
%
% License:
% CC BY-NC-SA 3.0 (http://creativecommons.org/licenses/by-nc-sa/3.0/)
%
%%%%%%%%%%%%%%%%%%%%%%%%%%%%%%%%%%%%%%%%%

%----------------------------------------------------------------------------------------
%	PACKAGES AND OTHER DOCUMENT CONFIGURATIONS
%----------------------------------------------------------------------------------------
\PassOptionsToPackage{fleqn}{amsmath}
\documentclass[paper=a4, fontsize=11pt]{scrartcl} % A4 paper and 11pt font size

\usepackage[T1]{fontenc} % Use 8-bit encoding that has 256 glyphs
\usepackage{fourier} % Use the Adobe Utopia font for the document - comment this line to return to the LaTeX default
\usepackage[english]{babel} % English language/hyphenation
\usepackage{amsmath,amsfonts,amsthm} % Math packages
\usepackage{graphicx}
\usepackage{lipsum} % Used for inserting dummy 'Lorem ipsum' text into the template

\usepackage{sectsty} % Allows customizing section commands
\allsectionsfont{\normalfont\scshape} % Make all sections centered, the default font and small caps

\usepackage{enumerate}
\usepackage{tikz}
\usepackage{tikz-qtree}
\usetikzlibrary{automata,positioning}
\usetikzlibrary{arrows,calc,fit}
\usepackage{latexsym}
\usepackage{float}
\usepackage{listings}
\usepackage{ulem}
\usepackage{amssymb}
\usepackage{longtable}
 \usepackage{array}

\usepackage[compact]{titlesec}
\titlespacing{\section}{0pt}{*10}{*1} % Spacing between sections

\usepackage{fancyhdr} % Custom headers and footers
\pagestyle{fancyplain} % Makes all pages in the document conform to the custom headers and footers
\fancyhead{} % No page header - if you want one, create it in the same way as the footers below
\fancyfoot[L]{} % Empty left footer
\fancyfoot[C]{} % Empty center footer
\fancyfoot[R]{\thepage} % Page numbering for right footer
\renewcommand{\headrulewidth}{0pt} % Remove header underlines
\renewcommand{\footrulewidth}{0pt} % Remove footer underlines
\setlength{\headheight}{0pt} % Customize the height of the header
\usepackage{fullpage}
\usepackage{hyperref}


\usepackage{amsmath}
\setlength{\mathindent}{25.0pt}

\numberwithin{equation}{section} % Number equations within sections (i.e. 1.1, 1.2, 2.1, 2.2 instead of 1, 2, 3, 4)
\numberwithin{figure}{section} % Number figures within sections (i.e. 1.1, 1.2, 2.1, 2.2 instead of 1, 2, 3, 4)
\numberwithin{table}{section} % Number tables within sections (i.e. 1.1, 1.2, 2.1, 2.2 instead of 1, 2, 3, 4)

\setlength\parindent{0pt} % Removes all indentation from paragraphs - comment this line for an assignment with lots of text
\def\ojoin{\setbox0=\hbox{$\Join$}%
  \rule[+0.095ex]{.3em}{.5pt}\llap{\rule[1.35ex]{0.3em}{0.5pt}}}
\def\leftouterjoin{\mathbin{\ojoin\mkern-5.8mu\bowtie}}
\def\rightouterjoin{\mathbin{\Join\mkern-5.8mu\ojoin}}
\def\fullouterjoin{\mathbin{\ojoin\mkern-5.8mu\Join\mkern-5.8mu\ojoin}}
%----------------------------------------------------------------------------------------
%	TITLE SECTION
%----------------------------------------------------------------------------------------

\newcommand{\horrule}[1]{\rule{\linewidth}{#1}} % Create horizontal rule command with 1 argument of height

\title{	
\normalfont \normalsize 
\textsc{Technische Universit\"at M\"unchen, Query Optimization} \\ [25pt] % Your university, school and/or department name(s)
\horrule{0.5pt} \\[0.4cm] % Thin top horizontal rule
\huge Exercise 8 \\ % The assignment title
\horrule{2pt} \\[0.5cm] % Thick bottom horizontal rule
}

\author{Matthias Gollwitzer, Jan Schalkamp} % Your name

\date{\normalsize\today} % Today's date or a custom date

\begin{document}
\lstset{language=SQL}
\maketitle % Print the title

%----------------------------------------------------------------------------------------
%	PROBLEM 1
%----------------------------------------------------------------------------------------

\section{Exercise 1}

\section{Exercise 2}

\section{Exercise 3}
Figure \ref{qg} depicts the chosen query graph, with relations cardinality is $\forall R_i: |R_i|=10$. Our rule set consists of the following rules: 
\begin{itemize}
	\item Commutativity:	$R_1 \Join R_2 \leadsto R_2 \Join R_1$
	\item Left Join Echange: $(R_1\Join R_2) \Join R_3 \leadsto (R_1\Join R_3) \Join R_2$
\end{itemize}
Furthermore Figures \ref{start solution} and \ref{optimalsolution} show the chosen start and the optimal solution. As both rules applied to the start solution yield no improvement, the optimal solution will never be found.

\begin{figure}[H]
	\centering
		\tikzstyle{b} = [text centered, thick]
		%\tikzstyle{c} = [rectangle, draw]
		\tikzstyle{l} = [draw, -latex',thick]  

\begin{tikzpicture}[-,auto,scale=1.25]
    \node [b]    (r1) {$R_1$};
    \node [b]    (r2)   at ([shift={(1,0)}] r1) {$R_2$};
    \node [b]    (r3)   at ([shift={(1,0.5)}] r2) {$R_3$};
    \node [b]    (r4)   at ([shift={(1,-0.5)}] r2) {$R_4$};
    
    
    
    \draw[-] (r1) edge node[below,pos=0.5,font={\sffamily\tiny}] {0.5}(r2);
    \draw[-] (r2) edge node[above,pos=0.5,font={\sffamily\tiny}] {0.5}(r3);
    \draw[-] (r2) edge node[below,pos=0.5,font={\sffamily\tiny}] {0.5} (r4);
    \draw[-] (r3) edge node[right,pos=0.5,font={\sffamily\tiny}] {0.1} (r4);

\end{tikzpicture}
	\caption{Example Query Graph}
	  \label{qg}
\end{figure}

\begin{figure}[H]
\centering
	\begin{tikzpicture}
	\tikzset{level distance=40pt, sibling distance=10pt}
	
	\Tree
	[.{$\Join (125)$}
		[.{$\Join (100)$} 
			[.{$\Join (100)$}
				[.{1} ]
				[.{4} ]
			]
			[.{3} ]
		]
		[.{2} ]
	]
	\end{tikzpicture}
\caption{Start solution}
  \label{startsolution}
\end{figure}

\begin{figure}[H]
\centering
	\begin{tikzpicture}
	\tikzset{level distance=40pt, sibling distance=10pt}
	
	\Tree
	[.{$\Join (125)$}
		[.{$\Join (25)$} 
			[.{$\Join (10)$}
				[.{3} ]
				[.{4} ]
			]
			[.{2} ]
		]
		[.{1} ]
	]
	\end{tikzpicture}
\caption{Optimal solution}
  \label{optimalsolution}
\end{figure}


\end{document}