%%%%%%%%%%%%%%%%%%%%%%%%%%%%%%%%%%%%%%%%%
% Short Sectioned Assignment
% LaTeX Template
% Version 1.0 (5/5/12)
%
% This template has been downloaded from:
% http://www.LaTeXTemplates.com
%
% Original author:
% Frits Wenneker (http://www.howtotex.com)
%
% License:
% CC BY-NC-SA 3.0 (http://creativecommons.org/licenses/by-nc-sa/3.0/)
%
%%%%%%%%%%%%%%%%%%%%%%%%%%%%%%%%%%%%%%%%%

%----------------------------------------------------------------------------------------
%	PACKAGES AND OTHER DOCUMENT CONFIGURATIONS
%----------------------------------------------------------------------------------------
\PassOptionsToPackage{fleqn}{amsmath}
\documentclass[paper=a4, fontsize=11pt]{scrartcl} % A4 paper and 11pt font size

\usepackage[T1]{fontenc} % Use 8-bit encoding that has 256 glyphs
\usepackage{fourier} % Use the Adobe Utopia font for the document - comment this line to return to the LaTeX default
\usepackage[english]{babel} % English language/hyphenation
\usepackage{amsmath,amsfonts,amsthm} % Math packages
\usepackage{graphicx}
\usepackage{lipsum} % Used for inserting dummy 'Lorem ipsum' text into the template

\usepackage{sectsty} % Allows customizing section commands
\allsectionsfont{\normalfont\scshape} % Make all sections centered, the default font and small caps

\usepackage{enumerate}
\usepackage{tikz}
\usepackage{tikz-qtree}
\usetikzlibrary{automata,positioning}
\usetikzlibrary{arrows,calc,fit}
\usepackage{latexsym}
\usepackage{float}
\usepackage{listings}
\usepackage{ulem}
\usepackage{amssymb}
\usepackage{longtable}
 \usepackage{array}

\usepackage[compact]{titlesec}
\titlespacing{\section}{0pt}{*10}{*1} % Spacing between sections

\usepackage{fancyhdr} % Custom headers and footers
\pagestyle{fancyplain} % Makes all pages in the document conform to the custom headers and footers
\fancyhead{} % No page header - if you want one, create it in the same way as the footers below
\fancyfoot[L]{} % Empty left footer
\fancyfoot[C]{} % Empty center footer
\fancyfoot[R]{\thepage} % Page numbering for right footer
\renewcommand{\headrulewidth}{0pt} % Remove header underlines
\renewcommand{\footrulewidth}{0pt} % Remove footer underlines
\setlength{\headheight}{0pt} % Customize the height of the header
\usepackage{fullpage}
\usepackage{hyperref}


\usepackage{amsmath}
\setlength{\mathindent}{25.0pt}

\numberwithin{equation}{section} % Number equations within sections (i.e. 1.1, 1.2, 2.1, 2.2 instead of 1, 2, 3, 4)
\numberwithin{figure}{section} % Number figures within sections (i.e. 1.1, 1.2, 2.1, 2.2 instead of 1, 2, 3, 4)
\numberwithin{table}{section} % Number tables within sections (i.e. 1.1, 1.2, 2.1, 2.2 instead of 1, 2, 3, 4)

\setlength\parindent{0pt} % Removes all indentation from paragraphs - comment this line for an assignment with lots of text
\def\ojoin{\setbox0=\hbox{$\Join$}%
  \rule[+0.095ex]{.3em}{.5pt}\llap{\rule[1.35ex]{0.3em}{0.5pt}}}
\def\leftouterjoin{\mathbin{\ojoin\mkern-5.8mu\bowtie}}
\def\rightouterjoin{\mathbin{\Join\mkern-5.8mu\ojoin}}
\def\fullouterjoin{\mathbin{\ojoin\mkern-5.8mu\Join\mkern-5.8mu\ojoin}}
%----------------------------------------------------------------------------------------
%	TITLE SECTION
%----------------------------------------------------------------------------------------

\newcommand{\horrule}[1]{\rule{\linewidth}{#1}} % Create horizontal rule command with 1 argument of height

\title{	
\normalfont \normalsize 
\textsc{Technische Universit\"at M\"unchen, Query Optimization} \\ [25pt] % Your university, school and/or department name(s)
\horrule{0.5pt} \\[0.4cm] % Thin top horizontal rule
\huge Exercise 7 \\ % The assignment title
\horrule{2pt} \\[0.5cm] % Thick bottom horizontal rule
}

\author{Matthias Gollwitzer, Jan Schalkamp} % Your name

\date{\normalsize\today} % Today's date or a custom date

\begin{document}
\lstset{language=SQL}
\maketitle % Print the title

%----------------------------------------------------------------------------------------
%	PROBLEM 1
%----------------------------------------------------------------------------------------

\section{Exercise 1}
Make as usual via \textit{make} command. TPCH data is again expected in \textit{data/tpch/tpch} with semicolons only as delimiter and no other occurrences of them.
\subsection{DPsub}
	\begin{itemize}
		\item Execute via \textbf{./bin/homework7-1}
		\item \textbf{src/compiler/strategies/DPsubStrategy:} Implements the DPsub algorithm. We are using the algorithm that was introduced in the lecture, which enumerates subsets efficiently. The whole algorithm is essentially a 1:1 conversion from the lecture slides. Only really different part is checking the estimated sizes of operators to be joined, as tinyDB only comes with a Hash Join - so we put the smaller operation left, and the bigger right. There is a crossproduct flag in the header file. It's currently set to true, as it should be beneficial to the query we are running.
		\item \textbf{src/compiler/querygraph/QueryGraph:} Added the method \textit{convertBitmapToSet()} in order to operate on sets represented each by a bitmap. Other methods added to check if a connection between two sets exists in the query graph.
	\end{itemize}

\subsection{DPsize}
	We did it anyway. 
	\begin{itemize}
		\item Execute via \textbf{./bin/homework7-1b}
		\item \textbf{src/compiler/strategies/DPsubStrategy:} Implements the DPsize algorithm as shown in the lecture. Here we combine the efficient subset enumeration with efficient  subset generation (using Gosper's hack\footnote{ \url{http://en.wikipedia.org/wiki/Combinatorial_number_system\#Applications} }).
	
	\end{itemize}

\section{Exercise 2}
	\begin{itemize}
		\item Execute via \textbf{./bin/homework7-2} (generates a pseudo-random join tree with 4 inner nodes)
		\item \textbf{src/compiler/randomized/DyckGenerator:} Essentially a 1:1 conversion from the lecture slides. We are not sure if it works for more  than 33 inner nodes. We tend to say no, as C(34) is bigger than a unsigned long long can store.
		\item \textbf{src/compiler/randomized/Binomial:} Class for calculating binomial coefficients utilising memoization.
	\end{itemize}

\end{document}